%%%% DRAFT - SIMPLE TEXT %%%%%%

%% This is intended to provide examples of how to do such things as
%% include tables, figures, and program listings, and adjust layout,
%% etc. within TeX and LaTeX. It is in no way a substitute for the
%% many helpful resources which may be found online.  A few such
%% references are listed at the end of the main part of this document.

%% Lines (or parts of lines) such as these following percent sign '%'
%% are COMMENTS which do not appear in the output typeset document

%%%%%%% USEPACKAGE commands:

%% The \usepackage commands in the first section below load additional
%% packages in LaTeX to allow some extra commands, mostly for
%% convenience. Not all those here are needed in this document, but it
%% usually does no harm to include them, and many others are
%% available.  For example:

%% amssymb,amsmath:  These are part of the American Mathematical
%%                   Society distribution - extremely useful for
%%                   mathematical typesetting.

%% enumerate:  Governs enumerated list using \item

%% verbatim:   Allows text file to be included as it is - see \verbatiminput

%% ifpdf:      See note on Graphics below

%% For more information, see references cited at end of this doument
%% such as http://tobi.oetiker.ch/lshort/lshort.pdf

%%%%%% Note about including GRAPHICS FILES:

%% Some TeX implementations allow users to generate PDF output
%% directly while others produce DVI files (for example the commands
%% pdflatex or latex on Linux). These are interchangeable in most
%% respects but differ in the graphics formats which they expect when
%% including figures.  In particular, implementations producing PDF
%% files generally need PDF graphics files, while implementations
%% producing DVI files generally need PostScript files.  Discussion of
%% the relative merits is beyond the intended scope of this document.

%% In some implementations, the implementation is clever enough to
%% choose the right graphics files. For others, a command \ifpdf
%% (enabled by \usepackage ifpdf) is available which can allow for
%% EITHER option, switching automatically to look for a named figure
%% in the correct format.  For most users this will not be needed as
%% they will just use one or other type of latex processing, and we
%% will not use this here.

%%%%%% PREAMBLE, AND START AND END:

%% The first few lines beginning \documentstyle are to set up some
%% optional preferences in place of the defaults

%% The main part of the document always begins with "\begin{document}"
%% and ends with "\end{document}".

\documentclass[11pt,a4paper]{article}

  \usepackage[hmargin={25mm,25mm},vmargin={25mm,25mm}]{geometry}

  \usepackage{verbatim, ifpdf}
  \usepackage{graphicx, color}

% \usepackage{amssymb, amsmath}
% \usepackage{enumerate}

% *********************************************************************
% The following lines should be ignored unless you know what you are
% doing, since the hyperref package is somewhat temperamental! 

  \IfFileExists{hyperref.sty}{%
    \ifpdf
      \usepackage[pdftex]{hyperref}
    \else
      \usepackage[dvips]{hyperref}
    \fi
    \definecolor{maroon}{RGB}{128,0,0}
    \hypersetup{pdfborder=0 0 0,colorlinks=true,plainpages=false,
                pageanchor=true,linkcolor=black,urlcolor=maroon}%
  }{%
    \providecommand{\url}[1]{\texttt{##1}}
    \providecommand{\href}[2]{##2\footnote{~See \texttt{##1}.}}%
  }

% End of hyperref setup.
% *********************************************************************

% The following optional parameters adjust paragraph spacing and
% indentation
  \setlength{\parskip}{1ex plus 0.5ex minus 0.2ex}
  \setlength{\parindent}{0pt}
  \addtolength{\skip\footins}{1.5 mm}

%%%%%  END OF PREAMBLE %%%%%%%%%%%%%%%%

%%%%%  THE MAIN TEXT STARTS HERE %%%%%%

\begin{document}


\begin{center}\LARGE\bf
  Comments on the Phylogeography Project
\end{center}

\section*{Git Repository}

The git repository is here: \url{https://github.com/AntanasKal/Phylogeography}

\begin{verbatim}
git clone https://github.com/AntanasKal/Phylogeography.git
\end{verbatim}

\section*{Installing Anaconda}

Some suggestions on installing Anaconda:

\begin{verbatim}
wget https://repo.continuum.io/archive/Anaconda3-5.1.0-Linux-x86_64.sh
#see https://www.anaconda.com/download/#linux
bash Anaconda3-5.1.0-Linux-x86_64.sh
#add to .bashrc, then log out and back in
\end{verbatim}

\section*{Python libraries}

Python libraries not included in Anaconda package that had to be installed:

\begin{itemize}
\item dendropy
\item ercs
\item discsim
\item pymc3 (only needed to analyse output data, namely find HPD interval)
\end{itemize}

\url{https://dendropy.org/}

\url{https://github.com/jeromekelleher/discsim}

\url{https://pypi.org/project/ercs/}

\section*{Installing BEAST}

In the cluster I installed BEAST so that I could run it from command line (by adding PATH variable in \textbf{./bashrc}) following this tutorial: \url{https://beast.community/install_on_unix}.

Then BEAST could be run with Python by:

\begin{verbatim}
beast -overwrite -seed 123456795 "beast_xml_file.xml"
\end{verbatim}

But it might be easier to run BEAST from jar file. Just copy the \textbf{beast.jar} to a convenient directory and run:

\begin{verbatim}
java -jar beast.jar -overwrite -seed 123456795 "beast_xml_file.xml"
\end{verbatim}

If needed, BEAGLE can also be installed, but I did not use it.

Note that Python code might need to be changed in order to run 


\section*{Building PHYREX}
To build PHYREX binary from source use the following code:
\begin{verbatim}
git clone https://github.com/stephaneguindon/phyml.git
cd phyml
sh ./autogen.sh
./configure --enable-phyrex
make clean
make
\end{verbatim}

Afterwards in \textbf{src/} folder there will be \textbf{phyrex} binary file which can be placed in different different directory and it can be run by:

\begin{verbatim}
./phyrex --xml=file_name.xml
\end{verbatim}

\clearpage

\section*{How to Use the Code}



\subsection*{simulation.py}


Script \textbf{simulation.py} is the main script of the code. It simulates the phylogeny of a tree and Brownian motion along it, subsamples the tree in 4 scenarios, generates the needed files for BEAST and launches BEAST on these 4 subsampling scenarios.

\begin{verbatim}
python simulation.py 
\end{verbatim}

Some of the parameters of the program:

\begin{itemize}
\item \textbf{-N}: number of simulations in sequence in one run.
\item \textbf{-jobi}: index of a job. It is needed if we want to do a number of simulations in parallel and keep the track of files. So if \textbf{jobi} is $j$ and \textbf{N} is $n$, the simulations will be done and files will be generated for indices
$$nj, \ nj+1, \ ... , \ n(j+1) - 1$$ 

Usually I set \textbf{N} to 1 and iterate \textbf{jobi} from 0 to 99, so that 100 simulations would be done.


\item \textbf{-dims}: number of dimensions (1 or 2) for which the random walk is generated (default: 2). \textbf{Note:} for 1 dimension some parts of code might not work so they need to be commented out. These are some parts of the code that generate some less important output.

\item \textbf{-treetype}: type of tree generated.

nuc - nonultrametric coalescent

uc - ultrametric coalescent

bd - birth-death tree

yule - Yule tree

(default is "yule")
\item \textbf{-mcmc}: MCMC chain length for BEAST (default is 5000)

\item \textbf{-{}-linux}: whether the program is run on Cluster (different console commands are used the for executing BEAST).

\item \textbf{-ntips}: number of tips for ultrametric coalecent, birth-death and Yule trees (default 100)

\item \textbf{-ntipspp}: number of tips per period for nonultrametric coalescent trees (default 20).

\item \textbf{-nps}: number of periods for nonultrametric coalescent trees (default 25).

\item \textbf{-{}-c\_beast}: whether the program should generate input files (not output) for corrected BEAST.


\end{itemize}

Other parameters could be seen by:

\begin{verbatim}
python simulation.py -h
\end{verbatim}


Shell script \textbf{run\_beast.sh} runs 100 simulations on Cluster in parallel.
\begin{verbatim}
for i in $(seq 0 99)
do bsub -o console_output/out_"$i".txt -e console_output/err_"$i".txt\
python simulation.py -jobi "$i" -N 1 -dims 2\
-treetype yule -ntips 2000 --linux -mcmc 10000 --c_beast
done
\end{verbatim}


These simulation should take not much longer than 10 minutes to run. To run corrected BEAST the script \textbf{launch\_corrected\_beast.py} is needed. It takes the files previously generated by \textbf{simulation.py} and runs BEAST on them. These runs could take about 10 hours. The script has these parameters:

\begin{itemize}
\item \textbf{-sample\_index}: index of the sampling scenario (usually from 1 to 4)
\item \textbf{-i}: index of the simulation (usually from 0 to 99)
\end{itemize}

The shell script \textbf{run\_c\_beast.sh} executes \textbf{launch\_corrected\_beast.py} on all the files generated.
\begin{verbatim}
for i in $(seq 0 99)
do bgadd -L 10 /c_beast/sim"$i"
bsub -g /c_beast/sim"$i" -o console_output/c_beast_out1_"$i".txt\
-e console_output/c_beast_err1_"$i".txt\ 
python launch_corrected_beast.py -index $i -sample_index 1
bsub -g /c_beast/sim"$i" -o console_output/c_beast_out2_"$i".txt\
-e console_output/c_beast_err2_"$i".txt\
python launch_corrected_beast.py -index $i -sample_index 2
bsub -g /c_beast/sim"$i" -o console_output/c_beast_out3_"$i".txt\
-e console_output/c_beast_err3_"$i".txt\
python launch_corrected_beast.py -index $i -sample_index 3
bsub -g /c_beast/sim"$i" -o console_output/c_beast_out4_"$i".txt\
-e console_output/c_beast_err4_"$i".txt\
python launch_corrected_beast.py -index $i -sample_index 4
done
\end{verbatim}

Other shell scripts that are used in a similar way are \textbf{discmodel/run\_discs.sh},  \textbf{launch\_phyrex.py}, \textbf{launch\_uncorrected\_beast.py}, and \textbf{make\_plots.py},

\subsection*{treegenerator.py}

Script \textbf{treegenerator.py} generates the trees in various scenarios.
\begin{itemize}
\item \textbf{treegenerator.generate\_ultrametric\_coalescent\_tree(num\_tips, lamb)}: returns ultrametric coalescent tree with \textbf{num\_tips} of leaves and coalescent rate \textbf{lamb}.

\item \textbf{treegenerator.generate\_yule\_tree(num\_tips, br)}: returns a Yule tree with \textbf{num\_tips} leaves and birthrate \textbf{br}.

\item \textbf{treegenerator.generate\_nonultrametric\_coalescent\_tree(num\_tips\_per\_period, num\_periods, period\_length, lamb)}: returns nonultrametric coalescent tree with \textbf{num\_periods} of periods, \textbf{num\_tips\_per\_period} of tips per period, \textbf{period\_length} of time between periods and coalescent rate \textbf{lamb}

\item \textbf{treegenerator.generate\_birthdeath\_tree(num\_extinct, br, dr)}: returns a subtree of a birth-death with leaves being the first \textbf{num\_extinct} extinct nodes. Birthrate is \textbf{br} and deathrate is \textbf{dr}.
\end{itemize}

Other functions in the script that are useful:

\begin{itemize}
\item \textbf{treegenerator.simulate\_brownian(t, sigma, dimension)}: simulates Brownian motion along the tree \textbf{t} for $\sigma$ = \textbf{sigma} and returns the tree with every node having attributes \textbf{X} (and \textbf{Y} if \textbf{dimension} is 2).
\item \textbf{treegenerator.calculate\_times(t)}: calculates times of each node (seed node has time 0) and returns the tree with each node having \textbf{time} attribute.

\end{itemize}


\subsection*{sampling.py}

Script \textbf{sampling.py} is used to sample the tree according to different scenarios. The 4 ones that were mainly used are here:

\begin{itemize}
\item \textbf{sampling.sample\_unbiased(tree, dimension=2, sample\_ratio=0.1)}: returns subtree of \textbf{tree} with \textbf{sample\_ratio} of tips taken uniformly at random.

\item \textbf{sampling.sample\_biased\_most\_central(t, dimension=2, sample\_ratio=0.1)}: returns the subtree with leaves that are closest to the centre.

\item \textbf{sampling.sample\_biased\_diagonal(tree, dimension=2, sample\_ratio=0.1)}: returns the subtree with leaves that are closest to the diagonal.

\item \textbf{sampling.sample\_biased\_extreme(tree, dimension=2, sample\_ratio=0.1)}: returns the tree with leaves that have the largest $x$ coordinate.
\end{itemize}

\subsection*{Running PHYREX}

Script \textbf{run\_phyrex.sh} runs PHYREX on the output files generated.

\begin{verbatim}
for i in $(seq 0 99)
do bgadd -L 10 /yule/phyrex"$i"
bsub -g /yule/phyrex"$i" -o console_output/phyrex_out1_"$i".txt\
-e console_output/phyrex_err1_"$i".txt\
./phyrex --xml=output/phyrex/sampled1/phyrex_input/phyrex"$i".xml
bsub -g /yule/phyrex"$i" -o console_output/phyrex_out2_"$i".txt\
-e console_output/phyrex_err2_"$i".txt\
./phyrex --xml=output/phyrex/sampled2/phyrex_input/phyrex"$i".xml
bsub -g /yule/phyrex"$i" -o console_output/phyrex_out3_"$i".txt\
-e console_output/phyrex_err3_"$i".txt\
./phyrex --xml=output/phyrex/sampled3/phyrex_input/phyrex"$i".xml
bsub -g /yule/phyrex"$i" -o console_output/phyrex_out4_"$i".txt\
-e console_output/phyrex_err4_"$i".txt\
./phyrex --xml=output/phyrex/sampled4/phyrex_input/phyrex"$i".xml
done
\end{verbatim}

\subsection*{Other Scripts}
Scripts \textbf{beastxmlwriter.py} and \textbf{phyrexxmlwriter.py} are used for writing BEAST and PHYREX input files respectively.


\section*{Output Folders}
Output is organised in this way: \textbf{output} folder has subfolders 
\textbf{beast}, \textbf{c\_beast}, \textbf{phyrex} which contain input and output files of BEAST, corrected BEAST and PHYREX. Each of these folders contains 4 folder for the results of 4 sampling scenarios. There are a few other folders, but they are not necessary.
The \textbf{.trees.txt} files take lots of space and only the root locations were important for us. Processed root locations are put into \textbf{root\_data/} folders and if I am downloading the data to my computer I usually delete the \textbf{.tree.txt} files by going to the folders containing them and executing this command:

\begin{verbatim}
find . -name "*.trees.txt" -type f -delete
\end{verbatim} 


Here is a rough scheme of the \textbf{output/} folder:
\begin{itemize}
\item \textbf{beast}
\begin{itemize}
\item \textbf{sampled1}
\begin{itemize}
\item \textbf{beast\_input}
\item \textbf{beast\_output}
\item \textbf{root\_data}
\end{itemize}
\item \textbf{sampled2}
\begin{itemize}
\item ...
\end{itemize}
\item \textbf{sampled3}
\begin{itemize}
\item ...
\end{itemize}
\item \textbf{sampled4}
\begin{itemize}
\item ...
\end{itemize}
\end{itemize}

\item \textbf{phyrex}
\begin{itemize}
\item \textbf{sampled1}
\begin{itemize}
\item \textbf{phyrex\_input}
\item \textbf{phyrex\_output}
\end{itemize}
\item ...

\end{itemize}

\item \textbf{c\_beast}
\begin{itemize}
\item ...
\end{itemize}





\end{itemize}

\clearpage

\section*{Discsim Scripts}

The Jupyter notebook "Phylogeography Output Analysis" contains tests for diffusion rate comparisons across models.



\subsection*{discs.py}

This script runs a discsim simulation, generates BEAST and PHYREX files for that and runs BEAST.
Currently it is set up in such a way that simulations indexed 0-99 are sampled from a square having opposite corners at $(25,\ 25)$ and $(75,\ 75)$, and simulations indexed 100-199 are sampled from a square having opposite corner at $(45,\ 45)$ and $(55,\ 55)$. It outputs the obtained root in folder \textbf{output/root\_data/}.



Shell script \textbf{run\_discs.sh} runs the 200 simulations:

\begin{verbatim}
bgadd -L 200 /discs
for i in $(seq 0 200)
do bsub -g /discs -o console_output/out_"$i".txt -e\
console_output/err_"$i".txt python discs.py -jobi "$i" -N 1
done
\end{verbatim}

Shell script \textbf{run\_phyrex.sh} runs PHYREX on all the simulations:

\begin{verbatim}
bgadd -L 200 /LV_phyrex
for i in $(seq 0 199)
do bsub -g /LV_phyrex -o console_output/phyrex_out_"$i".txt\
-e console_output/phyrex_err_"$i".txt\
./phyrex --xml=output/phyrex/LV/phyrex_input/phyrex"$i".xml
done  
\end{verbatim}



\subsection*{Other scripts}

Script \textbf{newick.py} for converting directed trees outputted by discsim to newick trees.
I edited this script after taking it from here:

\url{https://github.com/tyjo/newick.py/blob/master/newick.py}

Scripts \textbf{treegenerator.py} and \textbf{beastxmlwriter.py} in the folder are just copies of these scripts from Phylogeography folder.

Script \textbf{disc\_phyrexxmlwriter.py} is edited script \textbf{phyrexxmlwriter.py} to write  PHYREX xml scripts in this case.



\clearpage




\end{document}
